\documentclass[11pt]{article}

\usepackage{amsmath}
\usepackage{amsthm}
\usepackage{amssymb}
\usepackage{color}
\usepackage[margin=0.8in]{geometry}
\usepackage{url}
\usepackage{paralist}

\usepackage[utf8]{inputenc}
\usepackage[OT1]{fontenc}


% \renewcommand{\rmdefault}{phv} % Arial
% \renewcommand{\sfdefault}{phv} % Arial


\newcommand{\op}{\mathop{\text{op}}}

\title{Verifiable Neural-Symbolic Systems}
\date{}

\begin{document}

\maketitle

\section{What do you want to do?}

%* What’ s the hunch or big idea you can ’t get off your
%mind? 
%* Is there a problem space you want to explore?
%* A hypothesis you want to test? A new method, tool, or
%technology you want to create?
%* How does your proposal align with or challenge the
%assumptions of the Mathematics for Safe AI?
%* What will be the output of your effort?

We want to put forward formal methods for the principled
analysis of multi-agent systems comprising agents equipped
with neural network components. We target in particular (i)
the derivation of formal semantics for neural-symbolic
multi-agent systems, able to model the interactions of
possibly unbounded collections of agents with neural and/or
symbolic perception and control mechanisms, (ii) the
construction of a specification language to express the
properties of these systems, including properties pertaining
to high-level attitudes of agency, such as epistemic and
strategic properties, (iii) the computational complexity
analysis of the semantics and specifications from (i) and
(ii), (iv) the development of efficient
optimisation-based verification methods, (v) the building of
a verification tool that implements the methods from (iv)
and supports a dedicated programming language bases on
$\lambda$-calculus for the ({\bf insert here the
characteristics of the language}) description of systems and
correctness properties.

Our goals are fully aligned with the Mathematics for Safe AI
assumptions.  They recognise the increasing integration of
AI modules within networks of interconnected components and
provide methods for analysing their safety.  The methods are
formal and  can thus identify errors outside of testing
scenarios, as opposed to testing, where the safety of
systems can only be ascertained within testing scenarios.

Towards the development of the methods, we acknowledge the
challenges pertaining to the extension of formal methods to account
for neural-symbolic systems. Firstly, verification of
unbounded safety properties for neural systems is
undecidable. Secondly, verification of unbounded systems,
comprising arbitrarily many components is also undecidable.
Thirdly, verification of isolated neural network components
is NP-complete. In tandem with these negative results, we
aim to shed light on classes of systems can be effectively
verified, thereby providing design guidelines for verifiable
AI. For instance, verification for standalone neural
networks is hindered by big weighted sums and non-linear
activation functions that in practice can often be
simplified at no or little performance drops. 

\begin{itemize}
\item Put forward formal methods for the principled analysis of
  multi-agent systems comprising agents equipped with neural network
  components.
  
\item Design a (low-level, machine and human readable) specification
  language for describing neural-symbolic multi-agent systems and the
  properties they should satisfy.

\item Develop general purpose techniques for efficient verification of
  NS-MAS using automated theorem provers.
\end{itemize}

\section{Why is it of scientific or technological importance?}

%* If it works out, how would it disrupt the current frontier
%of knowledge or supersede the state-of-the-art?
%* What capabilities or future work could it unlock?
%* Why do you want to spend your time on this?

\begin{itemize}
\item Many modern complex systems involving AI-based components can be
  modelled as neural-symbolic multi-agent systems.
  
\item There is no standard specification language for verification of
  neural-symbolic MAS.

\item The existing tool VenMAS currently requires hardcoding the
  systems specs, this is time-consuming and error-prone.

\item The existing tool VenMAS encodes the verification problem into
  the MILP feasibility problem. The scalability is a problem.

\item This would facilitate the verification of real-world
  neural-symbolic multi-agent systems through the creation of
  declarative specifications.

\item Better scalability of verification of complex real-world systems.
\end{itemize}


\section{Why hasn't been done yet?}

%* Have recent developments only now made it possible?
%* Has the research community overlooked it? Perhaps no one
%has put the pieces together like this before?
%* How is your approach different or unusual compared to
%what’s being funded or developed elsewhere?


Neural-symbolic MAS are a relatively novel type of systems. Initial
work \cite{Akintunde+18,Akintunde+20,Akintunde+22} interprets
neural-symbolic MAS as Neural Interpreted Systems. In
\cite{KouvarosBB24}, we consider parameterised MAS, where the number
of agents is not fixed.

VenMAS \cite{Akintunde+22b} is a research prototype that verifies
neural-symbolic multi-agent systems by a reduction to the feasibility
problem of a mixed-integer linear program.  Currently, to verify such
a system against bounded temporal logic properties in VenMAS, the user
has to hardcode programmatically the system specification and the
temporal property.

There exist proper specification languages, such as Moxi for symbolic model
checking of finite- and infinite-state systems against temporal logic
properties~\cite{Rozier+2024}, ISPL (the interpreted systems
programming language) for purely symbolic multi-agent systems and
their verification properties in temporal logics~\cite{LomuscioQR17},
CyPhyML for cyber-physical systems~\cite{Simko+13}, and Vehicle for
verification of neural-symbolic programs against expressive properties
in dependently typed lambda-calculus \cite{Daggitt+24}. However, none
of these languages is fully equipped to describe neural-symbolic
multi-agent systems and the properties about their temporal evolution.


\section{How do you plan to make it happen?}

%* Where will you start and what steps will you take?
%* What are the technical challenges?
%* What does success look like to you? If there are ways to
%measure it, do you have early thoughts on how you ’ll do
%that?

% \paragraph{WP1} We will formal semantics for neural-symbolic
% multi-agent systems, able to model the interactions of possibly
% unbounded collections of agents with neural and/or symbolic perception
% and control mechanisms. The verification properties will be expressed
% in temporal, epistemic and/or strategy logics.

LanguagENeuralSystems

\paragraph{WP1} (Paulo, Michael, Marco) We will design a programming
language based on Vehicle, Moxi and ISPL for specifying neural MAS and
their properties, guided by the following desiderata:
\begin{itemize}
\item the syntax should be human and machine readable.  
\item the language should be based on a typed lambda-calculus language.
\item the language should be easily extendable to allow specification
  of various properties such as temporal logic, epistemic logic,
  strategic properties, etc.
\item the language should not be tightly coupled to a particular
  formalisation of a neural-symbolic system.
\end{itemize}

\paragraph{WP2} (Elena, Panagiotis, Marco) First, we will design an
abstract intermediate language for capturing the evolution of an
NS-MAS. This will be an internal representation and an abstraction of
the language of mixed-integer linear constraints that would allow us
easy manipulation as well as optimisation of sets of such
constraints. We anticipate the need for at least 3 kinds of
assertions:
\begin{inparaenum}[\it (i)]
\item a standard linear constraint `$\vec{a} \cdot \vec{x} \op b$',
  where $\vec{a}$ is a vector of reals, $\vec{x}$ is a vector of
  variables, $\op$ is one of $\leq$, $\geq$ or $=$, $b$ is a real;
\item a conditional assertion `if $\delta = v$, then $C$' asserting
  that the constraint $C$ is only enforced when the value of an
  integer-valued variable $\delta$ is $v$;
\item a bound constraint `$l \leq x \leq u$' asserting that the value
  of variable $x$ lies in the interval $[l,u]$ for $l\leq u$.
\end{inparaenum}
This language would be expressive enough so as to capture the
evolution of an NS-MAS while being `interpretable', so as to allow
optimisation.  For instance, assume that the translation from the
input spec generates the following constraints for a binary variable
$\delta$:
\[
  \begin{array}{l}
    \text{if }\delta = 1\text{, then }x = 1\\
    \text{if }\delta = 0\text{, then }x = 2
  \end{array}
\]
Furthermore, assume that the bounds of $\delta$ are calculated to be
$[1,1]$. In that case, the constraints can be simplified to $x=1$.

% Similarly to Lens,
Second, this intermediate language would be formulated as a dependent
type. We then will devise a proof theory so that translating
specifications to system evolutions encoded in the intermediate
language would amount to type inference.

\paragraph{WP3} (Paulo, Michael) We will extend VenMAS to take as
input a declarative specification of an NS-MAS and a property, and
verify the NS-MAS against that property. To this end, we will
implement a parser (T3.1 depends on WP1), and a translation from a
specification to an intermediate representation (T3.2 depends on WP2).

\paragraph{WP4} (Panagiotis, Elena) This workpackage is mainly
concerned with improving scalability of VenMAS. We will develop and
implement a compilation of an intermediate representation to a problem
that could be delegated to a backend. The existing backend of VenMAS
is a MILP solver that decides whether a mixed-integer linear program
is feasible (i.e., whether there is an assignment to the variables
that satisfies all constraints). As a first step, we will extend
VenMAS to include a compilation from an intermediate representation to
a mixed-integer linear program, which would serve as a baseline.
% a MILP feasibility problem
Then, to explore the possibility of using a more efficient backend
such as a neural network verifier optimised for GPU-based computation,
we will develop and implement a novel compilation of an intermediate
representation to a neural network verification problem.

As an example, consider again the constraints

\[
  \begin{array}{l}
    \text{if }\delta = 1\text{, then }x = 1\\
    \text{if }\delta = 0\text{, then }x = 2
  \end{array}
\]

This computation can be encoded as the following ReLU node:

\paragraph{WP5} (Paulo, Michael, Panagiotis, Elena) Use-cases and evaluation 

\begin{center}
  \begin{tabular}{ccccccccccccc}
    & 1 & 2 & 3& 4& 5 & 6 &7 &8& 9&10&11&12\\
    WP1& x & x & x \\
    WP2&   & x & x & x & x &   & \\
    WP3&   &   &   & x & x & x & x & x & \\
    WP4&   &   &   &   &   &   & x & x & x & x & x &   \\
    WP5&   &   &   &   &   &   &   &   &   & x & x & x \\
  \end{tabular}
\end{center}

\section{Why are you the right person (or
team) to work on it?}

%* Where does your insight come from?
%* Do you have any examples that showcase your motivation,
%niche expertise, or ability to commit?
%* If you're looking to extend your team, what key expertise
%should the next member bring?

Our team is uniquely placed to have the right expertise for this
project. Michael, Elena and Panagiotis have introduced neural-symbolic
multi-agent systems based on the Interpreted Systems semantics, and
published the VenMAS tool for verification of those against bounded
CTL and ATL properties.  Both Michael and Elena worked on the
implementation of VenMAS and a number of use-cases which have been
verified using VenMAS.  Panagiotis is an expert of symbolic MAS,
including parameterised systems where the exact number of the agents
is unknown.  Additionally, Elena and Panagiotis have an extensive
background in logic, formal verification of NNs and mixed-integer
linear solvers.

On the other hand, Marco will bring in his expertise in category
theory, theorem proving and typed programming language.  Paulo Torrens
is finishing PhD in theorem proving and CPS-calculus, and has relevant
theoretical and practical skills. Paulo will be the RA.

 

\bibliographystyle{plain}
\bibliography{bibl}

\end{document}
